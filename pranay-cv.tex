%% start of file `template.tex'.
%% Copyright 2006-2010 Xavier Danaux (xdanaux@gmail.com).
%% Copyright 2010-2011 Mark Liu (markwayneliu@gmail.com).
%
% This work may be distributed and/or modified under the
% conditions of the LaTeX Project Public License version 1.3c,
% available at http://www.latex-project.org/lppl/.

\documentclass[11pt,a4paper,sans]{moderncv}

\usepackage{verbatim}

% moderncv themes
\moderncvstyle{classic}
\moderncvcolor{blue}

% character encoding
\usepackage[utf8]{inputenc}                   % replace by the encoding you are using

% adjust the page margins
\usepackage[scale=0.8]{geometry}
%\setlength{\hintscolumnwidth}{3cm}						% if you want to change the width of the column with the dates
%\AtBeginDocument{\setlength{\maketitlenamewidth}{6cm}}  % only for the classic theme, if you want to change the width of your name placeholder (to leave more space for your address details
%\AtBeginDocument{\recomputelengths}                     % required when changes are made to page layout lengths


% personal data
\firstname{Pranay}
\familyname{Venkatesh}
%\address{(omitted for web)}{(omitted for web)}    % optional, remove the line if not wanted
%\mobile{(omitted for web)}                    % optional, remove the line if not wanted
\email{f20191004@pilani.bits-pilani.ac.in}                      % optional, remove the line if not wanted
\homepage{chemicalfiend.github.io}                % optional, remove the line if not wanted
%\extrainfo{\url{http://markliu.me}} % optional, remove the line if not wanted

% to show numerical labels in the bibliography; only useful if you make citations in your resume
%\makeatletter
%\renewcommand*{\bibliographyitemlabel}{\@biblabel{\arabic{enumiv}}}
%\makeatother

%\nopagenumbers{}                             % uncomment to suppress automatic page numbering for CVs longer than one page
%----------------------------------------------------------------------------------
%            content
%----------------------------------------------------------------------------------
\begin{document}
\maketitle

\section{Research Interest}

\cvline{}{I am interested in using theory and computational tools to model disordered materials and trying to develop the next generation of semiconductor optoelectronic devices.}

\section{Education}
\cventry{2019--2024}{M.Sc., Chemistry}{Birla Institute of Technology and Science}{Pilani, India}{}{}
\cventry{2019--2024}{B.E., Chemical Engineering}{Birla Institute of Technology and Science}{Pilani, India}{}{}

\section{Research Experience}

\cventry{summer 2022}{Research Intern}{Computational Materials Engineering Lab, Boise State University}{Boise, Idaho}{}{
    Mentor: Professor Eric Jankowski at the Micron School of Materials Science and Engineering, Boise State University at Idaho. 
    \begin{itemize}
        \item Analyzed Y6 and BTO materials and assessed their candidacy for organic photovoltaic applications.
        \item Estimated the force-field parameters of the molecules using MP4 and TD-DFT methods in Gaussian09 as implemented in QUBEKit.
        \item Determined the morphology and self-assembly of these materials using Molecular Dynamics (MD) simulations in HOOMD-Blue.
        \item Evaluated the charge-carrier transport properties and electron mobilities using semi-empirical quantum chemical calculations and kinetic monte carlo simulations.
        \item Presented a poster at ICUR '22 describing our workflow.
    \end{itemize}
}

\cventry{2021--2023}{Undergraduate Research Assistant}{Birla Institute of Technology and Science}{Pilani, India}{}{
    Mentor : Professor Sarbani Ghosh, Department of Chemical Engineering
    \begin{itemize}
        \item Worked on conducting polymer materials for optoelectronic and gas sensing applications.
        \item Devised new simulation setups for studying morphology of polymeric systems under nano-confinement.
        \item Obtained novel results regarding cylindrical nanoconfinement of PEDOT molecules by performing MD simulations in LAMMPS.
        \item Analyzed the self-assembly patterns of various polymers by writing bespoke computational codes in Julia.
    \end{itemize}
}

\cventry{summer 2021}{Summer Intern}{CSIR - Central Leather Research Institute}{Chennai, India}{}{
Analyzed collagen molecules using X-Ray Crystallography and identified key components of its structural biology using the XRD data.
Performed multiple sequence alignment studies on ClustalX to identify the patterns in protein sequences that are responsible for causing \emph{Osteogenesis imperfecta} in humans.
}

\section{Open Source Experience}

\cventry{summer 2022}{Julia Season of Code Contributor}{JuliaMolSim - Molly.jl}{}{}{
    \begin{itemize}
        \item Implemented bond and angle constraint algorithms in the Molly.jl framework.
        \item Coded in analysis features such as velocity autocorrelation.
        \item Implemented various interatomic potentials and bonded interactions that help model various systems.
    \end{itemize}
}

\section{Publications}

\cvline{1}{Sukanya Das, \textbf{Pranay Venkatesh}, Sarbani Ghosh, K S Narayan. Ordered and Disordered Microstructures of Confined Conducting Polymers - Submitted to Soft Matter}

\cvline{2}{\textbf{Pranay Venkatesh}, Eric Jankowski, Gwen White. Computational Challenges to Predicting Morphology of Large Macromolecule Blends, Poster at ICUR 2022}

\section{Teaching Experience}

\cventry{spring 2023}{Teaching Assistant}{Transport Phenomena}{}{}{
    \begin{itemize}
        \item Wrote comprehensive lecture notes for the course.
        \item Demonstrated some transport problems and their numerical solutions using simulations.
        \item Taught a session related to stream function and boundary layer theory in 2D flows.
    \end{itemize}
}


\section{Projects}

\cventry{fall 2022}{Molecular Dynamics Simulations of Liquids under Ultra-Confinement}{\break Study Project, Dept. of Chemistry}{BITS Pilani}{}{
    Evaluated the properties of nano-scale evaporation of Argon at low temperatures in the presence of a solid with Molecular Dynamics (MD) simulations. Determined the Hamaker coefficient, a metric for "wettability" in an Argon-Platinum system for various geometries.
}

\cventry{fall 2022}{Smart Insoles for Shock Absorption}{Course Project, Introduction to MEMS}{BITS Pilani}{}{
    Designed an insole with a smart electrorheological gel that works as a shock absorber in athlete's shoes, preventing damage and improving quality of sport. Modelled the gel and the effect of the shoe in COMSOL multiphysics.
}


\cventry{fall 2022}{Effective Models for Momentum Transport in Solid-Liquid Interfaces in MEMS Devices}{Course Project, Transport Phenomena}{BITS Pilani}{}{
    Obtained velocity profiles for a fluid in a nano-channel under various models. Leveraged direct-simulation monte carlo (DSMC) simulations to determine how slip in a solid-fluid boundary plays a large impact in micro and nano-scale systems.
}

\cventry{fall 2022}{Forced Convection Cooling of Integrated Circuits}{Course Project, Transport Phenomena}{BITS Pilani}{}{
    Demonstrated the harsh impact of IC heating on a PCB and designed a cooling solution that uses forced convection heat transfer using computational fluid dynamics (CFD) simulations in Ansys Icepak.  
}

\cventry{spring 2022}{Gallium Nitride Applications and Subsequent Improvements}{Course Project, Materials Science and Engineering}{BITS Pilani}{}{
    \begin{itemize}
        \item Reviewed the use of Gallium Nitride (GaN) in high-speed networks and power electronics. Interviewed prominent researchers in the field to understand the material processing and morphological features of its heterostructure.
        \item Surveyed diode and transistor device performance with the use of Gallium Nitride composites with NbN and other materials.
    \end{itemize}
}

\cventry{spring 2020}{Theoretical Investigation of Aggregation-Induced Emission (AIE) Molecules}{}{BITS Pilani}{}{
    Performed TD-DFT calculations with various functionals to reproduce the absorption and emission spectra for a novel Near IR (NIR) emissive molecule.
}

\section{Technical Experience}
\subsection{Extremely Proficient With}
\cvline{languages}{Julia, Python}
\cvline{technologies}{LAMMPS, HOOMD-Blue, \LaTeX{}, Bash Scripting, Git, Vim, Linux, Gnuplot}
\subsection{Have Experience With}
\cvline{languages}{Fortran-90, Java, Lua, C, MATLAB}
\cvline{technologies}{VMD, NWChem, Gaussian09, ANSYS Icepak, COMSOL, CUDA, Slurm}

\section{Conferences and Workshops}

\cvline{summer 2022}{JuliaCon 2022}

\cvline{summer 2022}{Idaho Conference on Undergraduate Research}

\cvline{summer 2021}{JuliaCon 2021}


\section{Relevant Coursework}
\cvline{core courses}{Materials Science and Engineering, Numerical Methods for Chemical Engineers, Physical Chemistry 2 (Quantum Mechanics), Physical Chemistry 3 (Group Theory and Many-electron Theory), Physical Chemistry 4 (Statistical Mechanics and Theories of Reaction Rates)}
\cvline{electives}{Electronic Correlation in Atoms and Molecules, Statistical Thermodynamics, Introduction to MEMS, Chemistry of Materials, Transport Phenomena, Quantum Information and Computing}
\cvline{online courses}{Introduction to Tensorflow, Improving Deep Neural Networks}


\section{Extracurriculars and Hobbies}

\cventry{}{BITS-ACM (Association of Computing Machinery)}{}{}{}{
    Promoted and motivated an understanding of scientific computing within the club. Wrote articles on simple machine learning and game development in python for the BITS-ACM blog.}

\cventry{}{The Eastern Outlook}{}{}{}{
    Eastern culture and anime club within the institute. Helped conduct events, games, discussions and expositions related to Japanese, Chinese and Korean art forms.}

\cventry{}{Other Hobbies}{}{}{}{
    Carnatic violin; Cricket; Philosophy; Electronics, operating systems and embedded software}

\end{document}
