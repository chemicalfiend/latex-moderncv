%% start of file `template.tex'.
%% Copyright 2006-2010 Xavier Danaux (xdanaux@gmail.com).
%% Copyright 2010-2011 Mark Liu (markwayneliu@gmail.com).
%
% This work may be distributed and/or modified under the
% conditions of the LaTeX Project Public License version 1.3c,
% available at http://www.latex-project.org/lppl/.

\documentclass[11pt,a4paper,sans]{moderncv}

\usepackage{verbatim}

% moderncv themes
\moderncvstyle{classic}
\moderncvcolor{blue}

% character encoding
\usepackage[utf8]{inputenc}                   % replace by the encoding you are using

% adjust the page margins
\usepackage[scale=0.8]{geometry}
%\setlength{\hintscolumnwidth}{3cm}						% if you want to change the width of the column with the dates
%\AtBeginDocument{\setlength{\maketitlenamewidth}{6cm}}  % only for the classic theme, if you want to change the width of your name placeholder (to leave more space for your address details
%\AtBeginDocument{\recomputelengths}                     % required when changes are made to page layout lengths

% personal data
%\firstname{\Large{Pranay}}
%\familyname{Venkatesh \\ Curriculum Vitae \\ PhD Chemistry \\ (UM ID : 95978225)}

\firstname{Pranay}
\familyname{Venkatesh}
%\address{(omitted for web)}{(omitted for web)}    % optional, remove the line if not wanted
%\mobile{(omitted for web)}                    % optional, remove the line if not wanted
\email{f20191004@pilani.bits-pilani.ac.in}                      % optional, remove the line if not wanted
\homepage{chemicalfiend.github.io}                % optional, remove the line if not wanted
%\extrainfo{\url{http://markliu.me}} % optional, remove the line if not wanted

% to show numerical labels in the bibliography; only useful if you make citations in your resume
%\makeatletter
%\renewcommand*{\bibliographyitemlabel}{\@biblabel{\arabic{enumiv}}}
%\makeatother

%\nopagenumbers{}                             % uncomment to suppress automatic page numbering for CVs longer than one page
%----------------------------------------------------------------------------------
%            content
%----------------------------------------------------------------------------------
\begin{document}
\maketitle

\section{Education}
\cventry{2024--2029}{PhD, Chemical Physics}{University of Colorado}{Boulder, Colorado}{}{}
\cventry{2019--2024}{M.Sc., Chemistry}{Birla Institute of Technology and Science}{Pilani, India}{}{}
\cventry{2019--2024}{B.E., Chemical Engineering}{Birla Institute of Technology and Science}{Pilani, India}{}{}

\section{Research Experience}

\cventry{2023--2024}{Imperial College London}{Dept of Chemistry}{London, England}{}{
    Mentor: Dr. Jarvist Moore Frost at Department of Chemistry, Imperial College London.
    \begin{itemize}
        \item Worked on using real-time Path Integrals to perform Non-Adiabatic Dynamics simulations of organic materials under photoexcitation.
        \item Developed workflows to model upconverting bilayer devices that have a high photoconversion efficiency as organic photovoltaics.
        \item Applied these methods to the Holstein Hamiltonian to study polaron mobility in organic crystals.
    \end{itemize}
}

\cventry{summer 2022}{CME Lab, Boise State University}{Micron School of MSE}{Boise, Idaho}{}{
    Mentor: Professor Eric Jankowski at the Micron School of Materials Science and Engineering, Boise State University.
    \begin{itemize}
        \item Analyzed Y6 and BTO materials and assessed their candidacy for organic photovoltaics.
        \item Estimated the force-field parameters of the molecules using MP4 and TD-DFT methods in Gaussian09 as implemented in QUBEKit.
        \item Determined the morphology and self-assembly of these materials using Molecular Dynamics (MD) simulations in HOOMD-Blue.
        \item Evaluated the charge-carrier mobilities using semi-empirical quantum chemical calculations and kinetic monte carlo simulations.
        \item Presented a poster at ICUR '22 describing our workflow.
    \end{itemize}
}

\cventry{2021--2023}{Computational Materials Design and Simulation Lab}{BITS Pilani}{Rajasthan, India}{}{
    Mentor : Professor Sarbani Ghosh, Department of Chemical Engineering
    \begin{itemize}
        \item Worked on conducting polymer materials for optoelectronics and gas sensing applications.
        \item Devised new simulation setups for studying morphology of nano-confined polymeric systems.
        \item Obtained novel results regarding cylindrical nanoconfinement of PEDOT molecules by performing MD simulations in LAMMPS.
        \item Analyzed the self-assembly patterns of various polymers by writing bespoke computational codes.
        \item Developed a scheme for inferring the charge carrier mobilities of various MD morphologies from
LAMMPS using Marcus Hopping Rate Theory and Kinetic Monte Carlo simulations.
    \end{itemize}
}

\cventry{summer 2021}{Central Leather Research Institute}{CSIR-CLRI}{Chennai, India}{}{
Analyzed collagen molecules using X-Ray Crystallography and identified key components of its structural biology using the XRD data.
Performed multiple sequence alignment studies on ClustalX to identify the patterns in protein sequences that are responsible for causing \emph{Osteogenesis imperfecta} in humans.
}


\section{Publications}

\cvline{1}{Sukanya Das, \textbf{Pranay Venkatesh}, Sarbani Ghosh, K S Narayan. Ordered and Disordered Microstructures of Nanoconfined Conducting Polymers, RSC Soft Matter 2023 doi : \href{https://doi.org/10.1039/D3SM00379E}{10.1039/D3SM00379E}}

\pagebreak

\section{Open Source Experience}

\cventry{summer 2022}{Julia Season of Code Contributor}{JuliaMolSim - Molly.jl}{}{}{
    \begin{itemize}
        \item Implemented bond and angle constraint algorithms in the Molly.jl framework.
        \item Coded in analysis features such as velocity autocorrelation.
        \item Implemented various interatomic potentials and bonded interactions for chemical systems.
        \item Presented my work at JuliaCon 2022.
    \end{itemize}
}



\section{Teaching Experience}

\cventry{spring 2023}{Transport Phenomena}{Course Teaching Assistant}{BITS Pilani}{}{
    \begin{itemize}
        \item Wrote comprehensive lecture notes for the course.
        \item Demonstrated some transport problems and their numerical solutions using simulations.
        \item Taught a session related to stream function and boundary layer theory in 2D flows.
    \end{itemize}
}

\cventry{July 2023}{Substitute Teacher in Chemistry}{NPS International, Chennai}{}{}{
    \begin{itemize}
        \item Taught classes in mole concept and solution chemistry for Grades 11 and 12 as a substitute teacher in NPS International, Chennai
    \end{itemize}
}

 \cventry{April 2020}{Online High School Chemistry Classes}{NPS International, Chennai}{}{}{
     \begin{itemize}
         \item Taught online video classes on solution chemistry for Grade 12 during COVID at NPS International.
     \end{itemize}
 }


\section{Projects}

\cventry{fall 2022}{Molecular Dynamics Simulations of Liquids under Ultra-Confinement}{\break Study Project, Dept. of Chemistry}{BITS Pilani}{}{
    \begin{itemize}
        \item Evaluated the properties of nano-scale evaporation of Argon at low temperatures in the presence of a solid with Molecular Dynamics (MD) simulations.
        \item Determined the Hamaker coefficient, a metric for "wettability" in an Argon-Platinum system for various geometries.
    \end{itemize}
}

\cventry{fall 2022}{Smart Insoles for Shock Absorption}{Course Project, Intro to MEMS}{BITS Pilani}{}{
    Designed an insole with a smart electrorheological gel that works as a shock absorber in athlete's shoes, preventing damage and improving quality of sport. Modelled the gel and the effect of the shoe in COMSOL multiphysics.
}


\cventry{fall 2022}{Effective Models for Momentum Transport in Solid-Liquid Interfaces in MEMS Devices}{Transport Phenomena}{BITS Pilani}{}{
    Obtained velocity profiles for a fluid in a nano-channel under various models. Leveraged direct-simulation monte carlo (DSMC) simulations to determine how slip in a solid-fluid boundary plays a large impact in micro and nano-scale systems.
}

\cventry{fall 2022}{Forced Convection Cooling of Integrated Circuits}{Transport Phenomena}{BITS Pilani}{}{
     Demonstrated the impact of IC heating on a PCB and designed a forced convection heat transfer cooling solution using computational fluid dynamics (CFD) simulations in Ansys Icepak 
}

\cventry{spring 2022}{Gallium Nitride Applications and Subsequent Improvements}{Course Project, Materials Science and Engineering}{BITS Pilani}{}{
    \begin{itemize}
        \item Reviewed the use of Gallium Nitride (GaN) in high-speed networks and power electronics. Interviewed prominent researchers in the field to understand the material processing and morphological features of its heterostructure.
        \item Surveyed diode and transistor device performance with the use of Gallium Nitride composites with NbN and other materials.
    \end{itemize}
}

\cventry{spring 2020}{Investigation of Aggregation-Induced Emission (AIE) Molecules}{}{BITS Pilani}{}{
    Performed TD-DFT calculations with various functionals to reproduce the absorption and emission spectra for a novel Near IR (NIR) emissive molecule.
}

\pagebreak

\section{Technical Skills}
\subsection{Extremely Proficient With}
\cvline{languages}{Julia, Python}
\cvline{technologies}{LAMMPS, HOOMD-Blue, \LaTeX{}, Bash Scripting, Git, Vim, Linux, Gnuplot}
\subsection{Have Experience With}
\cvline{languages}{Fortran-90, Java, Lua, C, MATLAB}
\cvline{technologies}{VMD, NWChem, Gaussian09, ANSYS Icepak, COMSOL, CUDA, Slurm}

\section{Presentations}

\cventry{spring 2024}{CECAM MD@60}{}{}{Bangalore, India}{
   Poster : "Fast and Numerically Accurate Path Integral Techniques for Simulating Nonadiabatic Molecular Dynamics in Organic Photovoltaic Materials"
}

\cvline{summer 2022}{JuliaCon 2022}

\cventry{summer 2022}{ICUR 22}{}{}{Boise, Idaho}{
   Poster : "Computational Challenges to Predicting Morphology of Large Macromolecule Blends"
}

\cvline{summer 2021}{JuliaCon 2021}


\section{Relevant Coursework}
\cvline{core courses}{Materials Science and Engineering, Numerical Methods for Chemical Engineers, Physical Chemistry 2 (Quantum Mechanics), Physical Chemistry 3 (Group Theory and Many-electron Theory), Physical Chemistry 4 (Statistical Mechanics and Theories of Reaction Rates)}
\cvline{electives}{Electronic Correlation in Atoms and Molecules, Statistical Thermodynamics, Introduction to MEMS, Chemistry of Materials, Transport Phenomena, Quantum Information and Computing}
\cvline{online courses}{Introduction to Tensorflow, Improving Deep Neural Networks}


\section{Extracurriculars and Hobbies}

\cventry{}{BITS-ACM (Association of Computing Machinery)}{}{}{}{
    Promoted and motivated an understanding of scientific computing within the club. Wrote articles on simple machine learning and game development in python for the BITS-ACM blog.}

\cventry{}{The Eastern Outlook}{}{}{}{
    Eastern culture and anime club within the institute. Helped conduct events, games, discussions and expositions related to Japanese, Chinese and Korean art forms.}

\cventry{}{Other Hobbies}{}{}{}{
    Sports; Philosophy; Cryptic Crossword; Carnatic Music}

\end{document}
